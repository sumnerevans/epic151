\documentclass[11pt]{article}
\usepackage[margin=1.0in]{geometry}
\usepackage{graphicx}

\title{User Empathy Reflection}
\author{Jonathan "Sumner" Evans}
\date{\today}

\begin{document}
\maketitle

\section{Introduction} % TODO: Finish Intro
Food deserts are areas that lack access to affordable fruits, vegetables, whole grains, lowfat milk,
and other foods that make up the full range of a healthy diet.\footnotemark[1] Many of the people in
food deserts do not own cars making it difficult to go to a supermarket. % TODO: Finish

\footnotetext[1]{"A Look Inside Food Deserts", cdc.gov, 2016. [Online]. Available:
    http://www.cdc.gov/features/FoodDeserts/index.html. [Accessed: 11-Sep-2016].}

\section{Anticipated Bugs}
Before embarking on the User Empathy Experience with my team, I compiled a list of "bugs" that I
anticipated my team would encounter while in the food desert. Below is that list with an explanation
of each item:

\begin{itemize}
    \item \textbf{Difficulty finding enough quality fruits and vegetables.} One of the main
        characteristics of food deserts is that it is difficult for people living within their
        boundaries to find fresh fruits and vegetables. Because of this, I anticipated that it would
        be difficult for my team to find enough quality fruits and vegetables.

    \item \textbf{Difficulty staying under the SNAP budget.} For this assignment, we attempted to
        simulate a family's experience in a food desert. One way we did this was by challenging
        ourselves to buy all of the food for our meal on a SNAP budget. SNAP is a USDA program which
        provides food assistance for people living in poverty. SNAP provides approximately \$4.32
        per person per day to its beneficiaries. For a family of five, that works out to be \$21.15
        for the entire family for the day. I anticipated that we would have problems staying under
        that budget. I made this judgment by looking at the receipt from a recent grocery trip where
        my family spent about \$177.00. Dividing this number out between the four people in my
        family, would result in about \$69.00 per person. \$69.00 divided by \$4.32 is about $16$
        meaning that we would have had to live solely off of the food we bought during that shopping
        trip for $16$ days. In my family, it is common to buy groceries approximately once a week so
        we would have definitely not have been able to eat well for all $16$ days.

    \item \textbf{Difficulty finding stores that accept EBT.} An Electronic Benefits Transfer (EBT)
        Card is the method by which SNAP funds are distributed. EBT is not accepted at every food
        retailer so I anticipated that we might have trouble finding a retailer that accepts EBT.

    \item \textbf{Difficulty finding low-fat protein sources.} Food deserts often have many fast
        food chains which server food that is high in fat content. Many times, these fast food
        chains are the only sources of protein in the area. % TODO: FINISH

    \item \textbf{Difficulty finding low-fat dairy.} Judging from my past experiences at gas station
        curb stores, high-fat dairy such as ice cream is in abundance, but low-fat dairy such as
        milk or yogurt is not as common. Since curb stores are often the only retailers of food in
        food deserts, I anticipated that it would be difficult to find such low-fat dairy.

    \item \textbf{Difficulties using public transportation.} I have used public transportation in
        many places around the world, with mixed results. In some cases, such as when I used the
        Tube in London, things went smoothly. In other cases, such as when I attempted to ride a bus
        in New York City, were not so successful. Despite having grown up in the Denver Metro area,
        I am not very experienced in using the public transportation so I decided to assume the
        worst case scenario. Judging from past mistakes while using public transportation, I thought
        that it would be likely that we might miss the bus, get on the wrong bus, have to wait a
        long time for the bus, or miss a connection with the Light Rail.

    \item \textbf{Difficulties carrying groceries on public transportation.} While in Boston for
        vacation this summer, I used the public transportation system to get around the city and I
        was often carrying a backpack. This was rather cumbersome at times. I thought that it would
        also be cumbersome to carry groceries.

    \item \textbf{Significant amount of time to find proper foods} I anticipated that it would take
        a significant amount of time to find the proper foods (if we were even able do to) because
        of all of the aforementioned factors.
\end{itemize}

\section{Actual Bugs}
During my teams trip to our assigned food desert, we encountered many difficulties. Some were
expected, but others were surprising and insightful.

\section{Bug Difference Analysis}

\section{Remaining Unknowns}

\section{Summary and Conclusion}

\end{document}
