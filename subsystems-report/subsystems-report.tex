\documentclass[12pt]{article}
\usepackage[margin=1.0in]{geometry}
\usepackage{graphicx}
\usepackage[pdftex]{hyperref}
\usepackage{wrapfig}
\usepackage{float}
\usepackage{enumitem}
\usepackage{textcomp}
\usepackage[english]{babel}
\usepackage{csquotes}
\usepackage[document]{ragged2e}
\usepackage[utf8]{inputenc}
\usepackage{indentfirst}
\usepackage{nicefrac}
\usepackage[dvipsnames]{xcolor}

\emergencystretch=\maxdimen
\hyphenpenalty=10000
\hbadness=10000
\MakeOuterQuote{"}
\renewcommand{\baselinestretch}{1.15}
\setlength\parindent{0.5in}
\setlength\RaggedRightParindent{\parindent}
\setlength\parskip{1em}
\let\supscr=\textsuperscript
\newcommand{\graytext}[1]{{\leavevmode\color{gray}#1}}

\begin{document}
\begin{titlepage}
    \centering
    \vspace{15cm}
    {\huge\bfseries Subsystems Report \par}
    \vspace{1cm}
    {\scshape\Large Jonathan Sumner Evans\par}
    \vfill
    {\scshape\large Section U\par}
    {\large \today\par}
    \vfill
\end{titlepage}

\tableofcontents

\graytext{
    \pagebreak
    \section{Introduction}
    A food desert is an area where access to fruits and vegetables is limited, too expensive, or
    nonexistent due to a lack of grocery stores and farmers markets within a convenient walking
    distance \cite{cdc-food-deserts}. People living in food deserts often rely on fringe food
    retailers and discount stores, such as gas stations and dollar stores, for food. These retailers
    tend to sell high-fat and processed foods which contributes to higher rates of obesity and
    diabetes in those food desert areas. According to the US Department of Agriculture, there is a
    food desert in the Wheat Ridge area located between Wadsworth, 32\supscr{nd} Avenue, and
    38\supscr{th} Avenue \cite{usda-food-deserts}.

    The team’s goal is to empower English and Spanish speaking Wheat Ridge residents over 65 years
    in age who live in a food desert and rely on food stamps to utilize a self-sustaining plant
    growing system. Most of the system will come prepackaged for safe installation and use. It will
    also partially use materials that can be sourced in the Wheat Ridge neighborhood. The net cost
    will be neutral or better after two years of plant harvest and will include features that
    consider the potential physical limitations of the stakeholders.

    \section{Solution Description}
    The team’s design addresses the problem with a system composed of five distinct and
    interconnected subsystems. The structural subsystem is a three-tiered structure that is compact,
    lightweight, and cost-effective so that the stakeholder can easily install and maintain it. With
    it’s 12 hole design and the reservoir at it’s base, the structure is the key connection between
    the hydroponic soda bottle system and the watering system.

    The watering system consists of two symbiotic subsystems: the reservoir and the cascading water
    system. The reservoir system uses four separate water tanks each of which correspond a growing
    stage in the plant cycle. The water is delivered to each plant by pumping water to the bottles
    in the top tier of the structure; the water then cascades down to the next two rows through
    tubing that connects the bottles together. Water drains from the lowest bottle back into the
    reservoir. This system only requires the stakeholder to refill the reservoirs one time per week.

    The nutrient subsystem is housed partially within the reservoir. Along with the structure and
    watering system, the stakeholder will receive packages of pre-measured nutrients that correspond
    to the growing cycle of the plants. By using a color-coding system, the nutrient system will be
    easy to understand regardless of the stakeholder’s native language. The stakeholder simply has
    to fill the reservoirs with the nutrients and water which will then be delivered to the plants
    via the water delivery system.

    Finally, the hydroponic (soilless) soda bottle system is composed of re-purposed 2-liter soda
    bottles and materials that can easily be sourced in the Wheat Ridge food desert. We chose a
    hydroponics system because they use up to 90\% less water, so they can be watered less often and
    are typified by less insects, a more controlled environment, and a steady harvest
    \cite{j-camas}. The soda bottles are the foundation for the snow peas, lettuce, and spinach to
    grow. The water-nutrient mixture is pumped into the bottles and is carried to the plant via a
    wick. This design prevents the stakeholder from overwatering and underwatering the plants and
    lowers the cost of the entire system by using easily-sourced substrate instead of soil.  The
    final solution also includes a tool which will allow the stakeholders to easily construct the
    bottle systems themselves despite their physical limitations. These five interconnected
    subsystems comprise the team's final solution (shown in Figure 1): an elegant, easy-to-maintain 
    food growing system designed with the stakeholders’ needs in mind.

    \pagebreak
    \begin{figure}[H]
        \centering
        \includegraphics[width=163mm]{resources/system-overview.jpg}
        \caption{System Sketch Showing Each Subsystem}
    \end{figure}
}

% ===== TODO =====
% - Check that the figure numbers are correct
% - Do a search for stakeholder
% - Do a search for subsystem
% - Three citations (1 done)
% - Three figures (2 done)

\section{Subsystem Description}

% Describe your subsystem functionality and key components.

The water delivery subsystem is the sole mechanism for transporting nutrients and water from
the reservoir to the soda bottles. This subsystem is critical to the overall system because if water
and nutrients are not delivered to the bottles, the plants cannot grow and no food can be produced.
The water delivery subsystem is comprised of four separate water circulation systems, one for each
nutrient stage. This helps facilitate the system's modular design. Each circulation system
corresponds to a single column of bottles and distributes water to each bottle in the column (see
Figure 1). Because we are using a wicking mechanism to transport water and nutrients from the bottom
of the bottle to the plant, the circulation system does not have to run continuously, it merely
needs to maintain the water level in each bottle. Because of this, our design requires the pumps to
be on a timer which allows the water to run for a few minutes every hour.

Since many of the water delivery system's components are interfaces with other subsystems, the only
component of the design is specific to the water deliver system: the tubing. The tubing carries
water and nutrients from each of the four reservoirs to the uppermost bottle. The tubing also
carries water from one bottle to the next in each column until the water drains out of the lowest
bottle in the column and back into the reservoir through another tube. In the final design,
$\nicefrac{1}{2}"$ tubing will be used. This diameter was chosen for two reasons:

\begin{enumerate}

    \item The fittings on the pumps that the team is considering are $\nicefrac{1}{2}''$ fittings.

    \item There are many options for $\nicefrac{1}{2}''$ bulkheads on the market. From my online
        searches, $\nicefrac{1}{4}"$ bulkheads are less common. These bulkheads are the primary
        interface point between the water delivery system and the hydroponic bottles (see Subsystem
        Interfaces for more details).

\end{enumerate}

% Physical properties, including dimensions, material type, specifications, weight, and
% construction.
% Include sketches with dimensions instead of long physical descriptions.

The tubing in the final product will be black to reduce the water's exposure to light. Exposure to
light causes algae growth \cite{doe-washington}. Since the team's goal is to prevent any algae
growth, this precaution is necessary.

As shown in Figure 2, there is a total of 3 feet of tubing between the pump and the uppermost
bottle. $7'\ \nicefrac{3}{4}''$ of the tubing is inside the reservoir and $2'\ 4\ \nicefrac{1}{4}''$
is outside the reservoir. Between each bottle is $1\ \nicefrac{1}{2}'$ of tubing ($\nicefrac{1}{2}'$
horizontally and $1'$ vertically).

\pagebreak
\begin{figure}[H]
    \centering
    \includegraphics[width=163mm]{resources/tubing.png}
    \caption{Dimensions of Tubing from Pump to Bottle}
\end{figure}

\section{Subsystem Interfaces}

% Explain the interfaces with the other subsystems: describe each interface and how your design
% addresses each. Reference a figure or a table.
There are two main interfaces between the water delivery system and the other subsystems:

\begin{enumerate}
    \item The connections to the four reservoirs in the reservoir system.
    \item The connections to and from the twelve hydroponic bottles.
\end{enumerate}

% Reservoir
The interface with the reservoir is twofold. First, the tubing of the water delivery subsystem
attaches to the pump in the reservoir. The tubing also passes through a hole cut in the ceiling of
the reservoir. To ensure that the pump and tubing interface well with one another,
$\nicefrac{1}{2}''$ tubing is being used. Additionally, the ceiling of the reservoir will have a
$\nicefrac{1}{2}''$ hole allowing the tube to pass through.

% Bottle connection
The interface with the bottle consists of a bulkhead fitting, the $\nicefrac{1}{2}''$ tubing and the
bottle itself. Figure 3 shows this interface. The bulkhead fitting has two parts: the bulkhead
body and the bulkhead lock nut. The bulkhead body and the threads are one piece and fit through a
hole in the soda bottle created using the bottle creating tool designed for the hydroponic bottle
system. The bulkhead lock nut is threaded and, once tightened, seals the gap between the bulkhead
body and the soda bottle. The threads of the bulkhead body also holds the tubing in place.

\begin{figure}[H]
    \centering
    \includegraphics[width=163mm]{resources/detail-a.png}
    \caption{Bulkhead Fitting Connected to Bottle and Tubing}
\end{figure}

\section{Subsystem Analysis}

% - Validation that it will work. If tested, what was the design of your experiment/test, and what
%   was the result?  If not tested, prove through research and analysis that your concept will
%   function as planned.

The water delivery subsystem is critical to the overall success of the team's system. Because of
this, extensive testing is required to ensure that the subsystem works. The four 

Because the seal between the tubing and the bottle is critical to the overall success of the team's
system, 



\section{Summary}

% THIS SHOULD BE THE FOCUS
% - Given your research, stakeholder feedback and/or testing, what are your recommendations for
%   design iterations and design implementation into the final, full-scale, production solution?

\pagebreak

\begin{thebibliography}{9}
    \bibitem{cdc-food-deserts}
    Centers for Disease Control and Prevention (2012).
    \textit{A Look Inside Food Deserts} [Online].
    Available: \url{http://www.cdc.gov/features/FoodDeserts/index.html}.
    [Accessed 11 Sept. 2016].

    \bibitem{usda-food-deserts}
    United States Department of Agriculture Economic Research Service (19 Oct. 2016).
    \textit{Food Access Research Atlas} [Online].
    Available:
    \url{http://www.ers.usda.gov/data-products/food-access-research-atlas/go-to-the-atlas.aspx}.
    [Accessed: 28 Oct. 2016].

    \bibitem{j-camas}
    J. Camas (04 Sept. 2014).
    \textit{Home Hydroponics} [Online].
    Available:
    \url{http://www.epicurious.com/archive/blogs/editor/2014/09/home-hydroponics-easy-tips-for-growing-fresh-produce-all-year.html}.
    [Accessed: Nov. 2, 2016].

    \bibitem{doe-washington}
    Washington State Department of Ecology.
    \textit{Algae Information} [Online].
    Available: \url{http://www.ecy.wa.gov/programs/wq/plants/algae/lakes/AlgaeInformation.html}.
    [Accessed: Oct. 19, 2016].

\end{thebibliography}

\end{document}
